%XeLaTeX
\documentclass{beamer}
\usetheme{focus}
\usepackage{multicol}
\title{Meeting--in--a--pdf}
%\subtitle{}
\author{}
\titlegraphic{\includegraphics[scale=.1]{logo.png}}
\institute{Formatted for screen sharing}
\date{}

\begin{document}
    \begin{frame}
        \maketitle
    \end{frame}
    
    \section*{Readings}
  
       \begin{frame}[plain]\large
       \begin{exampleblock}{The AA Preamble}
Alcoholics Anonymous is a fellowship of men and
women who share their experience, strength and
hope with each other that they may solve their
common problem and help others to recover from
alcoholism.
\bigskip

The only requirement for membership is a desire
to stop drinking. There are no dues or fees for
A.A. membership; we are self-supporting through
our own contributions. A.A. is not allied with any
sect, denomination, politics, organization or
institution; does not wish to engage in any
controversy; neither endorses nor opposes any
causes. Our primary purpose is to stay sober and
help other alcoholics to achieve sobriety.
\end{exampleblock}
    \end{frame}
   
\begin{frame}[noframenumbering,plain,allowframebreaks]%{How It Works}
\textcolor{example}{\textbf{How It Works}}
\bigskip

Rarely have we seen a person fail who has thoroughly followed our path. Those who do not recover are people who cannot or will  not  completely  give  themselves  to  this  simple  program,  usually  men  and  women  who  are  constitutionally  incapable  of  being honest with themselves. 
\bigskip

There  are  such  unfortunates.  They  are  not  at  fault;  they  seem  to  have  been  born  that  way.  They  are  naturally  incapable  of  grasping  and  developing  a  manner  of  living  which  demands  rigorous  honesty.  Their  chances  are  less  than  average.  There  are  those,  too,  who  suffer  from  grave  emotional  and  mental  disorders,  but  many  of  them  do  recover  if  they  have  the    capacity to be honest. 
\bigskip

Our  stories  disclose  in  a  general  way  what  we  used  to  be  like,  what happened, and what we are like now. If you have decided you  want  what  we  have  and  are  willing  to  go  to  any  length  to  get it, then you are ready to take certain steps. 
\bigskip

At  some  of  these  we  balked.  We  thought  we  could  find  an  easier, softer way. But we could not. With all the earnestness at our command, we beg of you to be fearless and thorough from the very start. Some of us have tried to hold on to our old ideas and the result was nil until we let go absolutely. 
\bigskip

Remember   that   we   deal   with   alcohol--cunning,   baffling,   powerful! Without help it is too much for us. But there is one who has all power--that one is god. May you find him now! 
\bigskip

Half  measures  availed  us  nothing.  We  stood  at  the  turning  point.   We   asked   his   protection   and   care   with   complete      abandon. 
\bigskip

Here are the steps we took, which are suggested as a program of recovery:  
\bigskip
%\end{frame}

 %   \begin{frame}[plain]%{The Twelve Steps}
    \footnotesize
%\textcolor{example}{The Twelve Steps}
\begin{enumerate}\setlength{\itemsep}{-2pt}
\item We admitted we were powerless over alcohol--that our lives had become unmanageable. \item   Came  to  believe  that  a  power  greater  than  ourselves  could restore us to sanity. \item  Made a decision to turn our will and our lives over to the care of god as we understood him.\item   Made   a   searching   and   fearless   moral   inventory   of      ourselves. \item   Admitted  to  god,  to  ourselves  and  to  another  human  being the exact nature of our wrongs. \item Were  entirely  ready  to  have  god  remove  all  these    defects of character. \item Humbly asked him to remove our shortcomings. \item   Made  a  list  of  all  persons  we  had  harmed,  and  became  willing to make amends to them all. \item Made  direct  amends  to  such  people  wherever  possible,  except when to do so would injure them or others. \item Continued to take personal inventory and when we were wrong promptly admitted it. \item Sought  through  prayer  and  meditation  to  improve  our  conscious  contact  with  god  as  we  understood  him,  praying  only  for  knowledge  of  his  will  for  us  and  the  power to carry that out. \item Having  had  a  spiritual  awakening  as  the  result  of  these  steps, we tried to carry this message to alcoholics and to practice these principles in all our affairs.
 \end{enumerate}
% \end{frame}
 
%\begin{frame}[plain]
%\textcolor{example}{How It Works, continued}
\normalsize
Many of us exclaimed, ``What an order! I can’t go through with it.'' 
\bigskip

 Do  not  be  discouraged.  No  one  among  us  has  been  able  to  maintain  anything  like  perfect  adherence  to  these  principles.  We  are  not  saints.  The  point  is  that  we  are  willing  to  grow  along  spiritual  lines.  The  principles  we  have  set  down  are  guides  to  progress.  We  claim  spiritual  progress  rather  than  spiritual perfection. 
\bigskip

Our  description  of  the  alcoholic,  the  chapter  to  the  agnostic,  and our personal adventures before and after make clear three pertinent ideas: 
\smallskip
\begin{itemize}
\item[a)] That  we  were  alcoholic  and  could  not  manage  our  own  lives.
\item[b)] That probably no human power could have relieved our alcoholism. 
\item[c)] That god could and would if he were sought. 
\end{itemize}
\end{frame}

\begin{frame}[plain]\scriptsize
%\begin{multicols}{2}
\textcolor{example}{\textbf{The Twelve Traditions}}
\begin{enumerate}% \setlength{\itemsep}{-2pt}
\item Our  common  welfare  should  come  first;  personal  recovery  depends upon AA unity. \item   For our group purpose there is but one ultimate authority--a loving  god  as  he  may  express  himself  in  our  group  conscience.  Our  leaders  are  but  trusted  servants;  they  do  not  govern.\item   The  only  requirement  for  AA  membership  is  a  desire  to  stop drinking. \item   Each group should be autonomous except in matters affecting other groups or AA as a whole. \item  Each  group  has  but  one  primary  purpose--to  carry  its  message to the alcoholic who still suffers. \item   An  AA  group  ought  never  endorse,  finance,  or  lend  the  AA  name  to  any  related  facility  or  outside  enterprise,  lest  problems  of  money,  property,  and  prestige  divert  us  from  our primary purpose. \item   Every AA group ought to be fully self-supporting, declining outside contributions. \item     Alcoholics     Anonymous     should     remain     forever          non-professional, but our service centers may employ special workers. \item    AA,  as  such,  ought  never  be  organized;  but  we  may  create  service  boards  or  committees  directly  responsible  to  those  they serve. \item  Alcoholics  Anonymous  has  no  opinion  on  outside  issues;  hence  the  AA  name  ought  never  be  drawn  into  public  controversy. \item Our public relations policy is based on attraction rather than promotion;  we  need  always  maintain  personal  anonymity  at  the level of press, radio, \& films. \item  Anonymity  is  the  spiritual  foundation  of  all  our  traditions,  ever reminding us to place principles before personalities. 
\end{enumerate}
%\end{multicols}
\end{frame}
 
 \begin{frame}[plain]
       \begin{exampleblock}{The 9\textsuperscript{th} Step Promises (from pages 83--84 in ``The Big Book'')}
 If we are painstaking about this phase of our
development, we will be amazed before we are
halfway through. We are going to know a new
freedom and a new happiness. We will not
regret the past nor wish to shut the door on it.
We will comprehend the word serenity and we
will know peace. No matter how far down the
scale we have gone, we will see how our
experience can benefit others. The feeling of
uselessness and self-pity will disappear. We will
lose interest in selfish things and gain interest in
our fellows. Self seeking will slip away. Our
whole attitude and outlook upon life will change.
Fear of people and of economic insecurity will
leave us. We will intuitively know how to handle
situations which used to baffle us. We will
suddenly realize that God is doing for us what
we could not do for ourselves.
\bigskip

Are these extravagant promises? We think not.
They are being fulfilled among us--sometimes
quickly, sometimes slowly. They will always
materialize if we work for them.
	\end{exampleblock}
\end{frame}
   
    \section*{Prayers}
        \begin{frame}[plain]\LARGE
        \begin{alertblock}{Serenity Prayer}
God, \\grant me the Serenity,\\ to accept the things I cannot change,\\ Courage to change the things I can, \\and the Wisdom to know the difference.
\end{alertblock}
    \end{frame}
    
            \begin{frame}[plain]\LARGE
        \begin{alertblock}{Third Step Prayer}
God, I offer myself to Thee--to build with me and to do with me as Thou wilt. Relieve me of the bondage of self, that I may better do Thy will. Take away my difficulties, that victory over them may bear witness to those I would help of Thy Power, Thy Love, and Thy Way of Life.  May I do Thy will always!
\end{alertblock}
    \end{frame}
    
                \begin{frame}[plain]\LARGE
        \begin{alertblock}{Seventh Step Prayer}
My Creator, I am now willing that you should have all of me, good and bad. I pray that you now remove from me every single defect of character which stands in the way of my usefulness to you and my fellows. Grant me strength, as I go out from here, to do your bidding.  Amen.
\end{alertblock}
    \end{frame}
    
                    \begin{frame}[plain]\large
        \begin{alertblock}{Eleventh Step Prayer (St. Francis Prayer)}
    Lord, make me a channel of Thy peace; that where there is hatred, I may bring love; that where there is wrong, I may bring the spirit of forgiveness; that where there is discord, I may bring harmony; that where there is error, I may bring truth; that where there is doubt, I may bring faith; that where there is despair, I may bring hope; that where there are shadows, I may bring light. that where there is sadness, I may bring joy. Lord, grant that I may seek rather to comfort, than to be comforted; to understand, than to be understood; to love, than to be loved.  For it is by self--forgetting, that one finds. It is by forgiving, that one is forgiven. It is by dying, that one awakens to Eternal Life. Amen.
\end{alertblock}
    \end{frame}
    
    \section*{Topics}
    \begin{frame}[noframenumbering, allowframebreaks]{Topics in Big Book (4th ed.)}
 %   \begin{multicols}{2}
\begin{itemize}%\setlength\itemsep{-.5em}
\item Acceptance – 14, 30, 207, 417, 420 \item Admission – 25, 72-73 \item Aloneness – 17, 89 \item Ambition – 68, 72, 77, 127, 129 \item Amends – 77, 82-83 \item Anger – 60, 61, 64, 66, 67, 111 \item Arrogance – 60, 61 \item Character Defects – 26, 69 \item Complacency – 82 \item Compassion – 108 \item Courage – 67, 68 \item Depression – 15, 67-68 \item Disease – 21, 23, 416 \item Easy Does It –  135 \item Envy – 68, 77 \item Faith – 14, 15, 48, 49, 52, 55\item Family Relationships – 68, 83, 97, 99, 100, 135\item Fear – 67-68, 115, 116 \item Freedom – 83, 84, 93, 133, 151, 552\item Financial – 98, 127\item Forgiveness – 70, 80 \item Gratitude – 132 \item Growth – 33, 63 \item Happiness – 17, 128-129, 132-133, 151\item Higher Power – 12, 28, 30, 44-49, 51-53, 55, 62-63, 93, 98, 100, 130, 164\item Honesty – 58, 64, 67, 70, 72-73, 82-83, 115, 212, 549\item Hope - 44, 45, 73, 163\item H.O.W. - 549\item Humility – 12, 13, 25, 63, 72-73, 93, 100, 218 \item Identification – 17, 93\item Illness – 22, 23, 30, 84-85, 133, 151\item Inventory – 25, 64-65, 69, 72-73, 86, 99-100, 126\item Insanity – 30, 37, 38, 57, 124\item Jealousy – 82, 119, 131\item Meditation – 86-87, 164\item Membership – 28\item Newcomers – 83, 93, 96-97, 128-129, 135, 164 
\item Open Mind – 12, 46-49, 51, 55, 62 \item Perfection – 60, 123, 126, 127, 135\item Prayer – 63, 66, 67, 70, 75, 76, 79, 80, 82, 85-87, 164, 215, 552\item Patience – 67-70, 82, 90, 98, 111, 118, 123, 126 \item Promises – 63, 75, 83-84, 100, 115-116, 120\item Rationalization – 64-65, 99-100, 550\item Recovery – 126, 127, 164\item Reprieve, Daily – 85 \item Resentment – 64, 552\item Sanity – 22-23, 84-85, 551\item Selfishness – 62\item Self-knowledge – 7, 36\item Self-pity – 60-61\item Self-Will – 60, 62, 84\item Serenity – 63-64, 68, 553, 551\item Service – 14-15, 77, 101\item Sex – 68-69, 70, 83, 99-100, 124, 134\item Slips – 70, 72-73, 120, 139\item Slogans - 135\item Sponsorship – 15-18, 25, 58, 88, 89, 98-100\item Spiritual Living – 46, 51, 60, 66, 83, 85, 97, 100, 101, 127, 135, 164\item Spiritual principles – 42, 47, 83, 93, 97, 116, 156 \item Success – 127\item Surrender – 33, 48, 58, 84-85, 100, 133, 151\item Steps – 8, 12, 13, 14, 59-89, 263 \item Temptation – 14-15, 85, 100-101\item Tolerance – 28, 66-67, 70\item Trouble – 35, 55, 131, 133\item Trust God – 98\item Twenty four hours – 16, 86\item Understanding – 568\item Unity – 17, 25\item  Weak – 20, 72, 115, 120, 154\item Will – 44-45, 48, 52-53, 55, 60-63, 93\item Willingness – 12-13, 26, 36-37, 41-42, 46-47, 53, 57, 60, 69, 70, 76, 77, 79, 93, 118, 124, 152, 153, 158-159, 162, 207, 214, 550  
\end{itemize}
%\end{multicols}
\end{frame}
%    \begin{frame}[plain]{Plain frame}
%        This is a frame with plain style and it is numbered.
%    \end{frame}

%    \begin{frame}[plain]{Plain frame}
%        This is a frame with plain style and it is numbered.
%    \end{frame}

%    \begin{frame}[plain]{Plain frame}
%        This is a frame with plain style and it is numbered.
%    \end{frame}
    
    
  
  
  
  \end{document}
  
  % Full instructions available at:
% https://github.com/elauksap/focus-beamertheme

    % Use starred version (e.g. \section*{Section name})
    % to disable (sub)section page.
%    \section{Section 1}
%    \subsection{Subsection 1.1}
%    \begin{frame}{Simple frame}
%        This is a simple frame.
%    \end{frame}
%
%    \begin{frame}[plain]{Plain frame}
%        This is a frame with plain style and it is numbered.
%    \end{frame}
%    
%    \subsection{Subsection 1.2}
%    \begin{frame}[t]
%        This frame has an empty title and is aligned to top.
%    \end{frame}
%    
%    \begin{frame}[noframenumbering]{No frame numbering}
%        This frame is not numbered and is citing reference \cite{knuth74}.
%    \end{frame}
%    
%    \begin{frame}{Typesetting and Math}
%        The packages \texttt{inputenc} and \texttt{FiraSans}\footnote{\url{https://fonts.google.com/specimen/Fira+Sans}}\textsuperscript{,}\footnote{\url{http://mozilla.github.io/Fira/}} are used to properly set the main fonts.
%        \vfill
%        This theme provides styling commands to typeset \emph{emphasized}, \alert{alerted}, \textbf{bold}, \textcolor{example}{example text}, \dots
%        \vfill
%        \texttt{FiraSans} also provides support for mathematical symbols:
%        \begin{equation*}
%            e^{i\pi} + 1 = 0.
%        \end{equation*}
%    \end{frame}
%
%    \section{Section 2}
%    \begin{frame}{Blocks}
%        \begin{block}{Block}
%            Text.
%        \end{block}
%        \pause
%        \begin{alertblock}{Alert block}
%            Alert \alert{text}.
%        \end{alertblock}
%        \pause
%        \begin{exampleblock}{Example block}
%            Example \textcolor{example}{text}.
%        \end{exampleblock}
%    \end{frame}
%    
%    \begin{frame}{Lists}
%        \begin{columns}[t, onlytextwidth]
%            \column{0.33\textwidth}
%                Items:
%                \begin{itemize}
%                    \item Item 1
%                    \begin{itemize}
%                        \item Subitem 1.1
%                        \item Subitem 1.2
%                    \end{itemize}
%                    \item Item 2
%                    \item Item 3
%                \end{itemize}
%            
%            \column{0.33\textwidth}
%                Enumerations:
%                \begin{enumerate}
%                    \item First
%                    \item Second
%                    \begin{enumerate}
%                        \item Sub-first
%                        \item Sub-second
%                    \end{enumerate}
%                    \item Third
%                \end{enumerate}
%            
%            \column{0.33\textwidth}
%                Descriptions:
%                \begin{description}
%                    \item[First] Yes.
%                    \item[Second] No.
%                \end{description}
%        \end{columns}
%    \end{frame}
%\setbeamertemplate{caption}[numbered]
%    \begin{frame}{Figures and Tables}
%        \begin{columns}
%            \column{0.4\textwidth}
%                \begin{figure}
%                    \centering
%                    \includegraphics{focuslogo.pdf}
%                    \caption{Figure caption.}
%                    \label{fig:focuslogo}
%                \end{figure}
%                
%            \column{0.6\textwidth}
%                \begin{table}
%                    \centering
%                    \begin{tabular}{rcc}
%                         & Heading 1 & Heading 2 \\\hline
%                        Row 1 & \(v_{11}\) & \(v_{12}\) \\
%                        Row 2 & \(v_{21}\) & \(v_{22}\) \\
%                        Row 3 & \(v_{31}\) & \(v_{32}\) \\
%                    \end{tabular}
%                    \caption{Table caption.}
%                    \label{tab:demo}
%                \end{table}
%        \end{columns}
%    \end{frame}
%    
%    \begin{frame}[focus]
%        Thanks for using \textbf{Focus}!
%    \end{frame}
%    
%    \appendix
%    \begin{frame}{References}
%        \nocite{*}
%        \bibliography{demo_bibliography}
%        \bibliographystyle{plain}
%    \end{frame}
%    
%    \begin{frame}{Backup frame}
%        \usebeamercolor[fg]{normal text}
%        This is a backup frame, useful to include additional material for questions from the audience.
%        \vfill
%        The package \texttt{appendixnumberbeamer} is used not to number appendix frames.
%    \end{frame}
%
